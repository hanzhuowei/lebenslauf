%%%%%%%%%%%%%%%%%%%%%%%%%%%%%%%%%%%%%%%%%
% Medium Length Professional CV
% LaTeX Template
% Version 2.0 (8/5/13)
%
% This template has been downloaded from:
% http://www.LaTeXTemplates.com
%
% Original author:
% Trey Hunner (http://www.treyhunner.com/)
%
% Important note:
% This template requires the resume.cls file to be in the same directory as the
% .tex file. The resume.cls file provides the resume style used for structuring the
% document.
%
%%%%%%%%%%%%%%%%%%%%%%%%%%%%%%%%%%%%%%%%%

%----------------------------------------------------------------------------------------
%	PACKAGES AND OTHER DOCUMENT CONFIGURATIONS
%----------------------------------------------------------------------------------------

\documentclass{resume} % Use the custom resume.cls style

\usepackage[left=0.75in,top=0.6in,right=0.75in,bottom=0.6in]{geometry} % Document margins

\name{Zhuowei Han} % Your name
\address{Pfaffenwaldring 44D, 70569, Stuttgart} % Your secondary addess (optional)
\address{+49/(0)176-61891464 or hanzhuowei1226@gmail.com} % Your phone number and email


\begin{document}

%----------------------------------------------------------------------------------------
%	OBJECTIVE SECTION
%----------------------------------------------------------------------------------------
 
\begin{rSection}{OBJECTIVE} 

xxx

\end{rSection}


%----------------------------------------------------------------------------------------
%	Summary 
%----------------------------------------------------------------------------------------
 
\begin{rSection}{SUMMARY OF QUALIFICATIONS}  
\begin{itemize}
\item Extensive background in signal processing (radar, ultrasonic sensor)
\item Skilled in algorithm design, analysis and optimization techniques. In particular, simulation and modeling experience with restricted Boltzmann Machine.
\item Detail-oriented and passionate self-starter, team-player with expertise in research, programming and troubleshooting.
\end{itemize}
\end{rSection}

%----------------------------------------------------------------------------------------
%	WORK EXPERIENCE SECTION
%----------------------------------------------------------------------------------------

\begin{rSection}{RESEARCH  EXPERIENCE}

\begin{rSubsection}{Master Thesis,  Speech Emotion Recognition with Deep Network
}{08/2014-Present}{}{Institute of Signal Processing and System Theory, University of Stuttgart}

\item Speech emotion signal processing to extract MFCC features
% \item Evaluated spectrum features of speech signals based on different database.
% \item Intergrated sparsity regularization to current the deep network to prevent overfitting.
\item Building probabilistic graphical model (Conditional RBM) for unsupervised feature learning, optimizing network parameters
\item 
% \item Fixed bugs in current Framework (Python).

\end{rSubsection}

%------------------------------------------------

\begin{rSubsection}{Study Thesis, Optimization and Validation of Adaptive Threshold Parameters in Ultra-Sonic Object Detection System 
}{09/2013-03/2014}{}{Robert Bosch GmbH}

\item Learned state of the art object detection techniques with ultra-sonic sensors in Driver Assistent System.
% \item Studied  based on ultra-sonic sensor.
\item Taken measurements with ultra-sonic sensors in rear bumper on different driving and parking grounds and collected sensor data.
\item Pre-processing measurement data within Matlab and comparing the influence on detection threshold with respect to individual signal feature.
\item Evaluated sending patterns of ultra-sonic sensor for near, mid and far distance of detection range as well as for different ground surfaces.
\item Optimized and validated the adaptive threshold parameters with CA-CFAR algorithm.
\item Implemented VBA code for integrating Excel data into current Matlab GUI analyse-tools, making analyse automated. 
% \item Performed strong and weak scalability test for the paralleled anelastic 3D wave propagation package(AWP) on Triton Shared Computing Cluster(TSCC). 
% \item Completed the performance(Profiling and tracing) analysis of AWP, using TAU on Kraken Cluster. 
% %\item Conducted large-scale 3D numerical simulations of seismic wave propagation on the distributed architecture of TSCC.
% %\item Established the file I/O interface between AWP and the full-3D waveform tomography package(F3DT) to improve run-time efficiency. 
% \item Researched the theory of Full 3D Waveform Tomography and documented the workflow to compile and run F3DT package on Triton Cluster.
\end{rSubsection}

%------------------------------------------------

\begin{rSubsection}{Practical Lab, Statistical Signal Processing – Automotive Radar
}{09/2013-01/2014}{Student Team Member}{Institute of Signal Processing and System Theory, University of Stuttgart}

\item Obtained basic concept of LFMCW-Radar for range and angle detection.
\item Pre-processing on raw radar signal and implemted adaptive threshold based on CA/OS-CFAR technique for range detection.
\item Implemented simple Kalman filter for object tracking. 
% \item Participated onshore seismic survey and electromagnetic data acquisition, improved team working and communication skills.
% \item Studied seismic interferometry and the theory of seismic imaging.

\end{rSubsection}

\begin{rSubsection}{Scientific Assistant, Implemtation in Matlab and VBA}{04/2013-07/2013}{Student Employee}{Institute of High-Frequency Technology, University of Stuttgart}

\item Implemented analyse-tool with Matlab GUI for antenna radiation pattern 
\item Data processing with Excel VBA.
% \item Studied the theory of frequency-domain Full Waveform Inversion. 

\end{rSubsection}
\end{rSection}


%----------------------------------------------------------------------------------------
%	EDUCATION SECTION
%----------------------------------------------------------------------------------------
\begin{rSection}{EDUCATION}
\begin{tabular}{l l}
 
{\sl M.S.,} & Electrical Engineering and Information Technology\\
\end{tabular}

University of Stuttgart, Germany. \hfill 2012-Present \\

\begin{tabular}{l l}
{\sl B.S.,} & Electronic and Information Engineering\\
\end{tabular}

University of Electronic Science and Technology at Xi'an, China. \hfill  2008-2012\\



\end{rSection}

%----------------------------------------------------------------------------------------
%	COMPUTER SKILLS SECTION
%----------------------------------------------------------------------------------------
\begin{rSection}{COMPUTER SKILLS}
\begin{tabular}{l l}
{\sl Languages:} &Python, \LaTeX{}, VBA. \\
{\sl Tools:} & Matlab (incl. GUI), MS-Office.
\end{tabular}

\end{rSection}

%-------------------------------Languages
%----------------------------------------------------------------------------------------
\begin{rSection}{Language Skills}
\begin{tabular}{l l l}
Chinese & Mother Spoke  \\
German  & Test-DaF 17 \\
English & CET-6
\end{tabular}
\end{rSection}

%----------------------------------------------------------------------------------------



%----------------------------------------------------------------------------------------
%	EXTRACURRICULAR ACTIVITIES SECTION
%----------------------------------------------------------------------------------------
\begin{rSection}{EXTRACURRICULAR  ACTIVITIES}

Rohde\&Schwarz Case Study 2013, Participant. \hfill 05/2013 \\
Intercultural Mentoring Program, University of Stuttgart, Mentee. \hfill 2012-2013 \\
\end{rSection}

%----------------------------------------------------------------------------------------
%	HONORS AND  AWARDS
%----------------------------------------------------------------------------------------
% \begin{rSection}{HONORS AND  AWARDS}
% 
% 
% \end{rSection}


\end{document}
